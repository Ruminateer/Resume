% 2019 Winter
\newcommand{\Pager}{
    \poswithprd{分页内存管理器}{2019.03}
    \begin{miniItemize}
        \item
        领导2人小组. 基于仅支持read\_enable与write\_enable的模拟MMU实现分页内存管理器. 管理器通过页缺失追踪更新物理分页的信息.
        \item
        支持多进程切换与fork新建进程, 通过实现写入时复制以提高性能. 支持动态内存分页与文件分页, 实现时钟算法以管理物理内存.
        \item
        实现面向对象编程. 通过控制模块间耦合度, 提升代码可读性, 降低代码调试维护难度.
    \end{miniItemize}
}

\newcommand{\ThreadLib}{
    \poswithprd{线程库}{2019.02}
    \begin{miniItemize}
        \item
        领导3人小组. 基于ucontext在多核处理器上实现thread, mutex, 与 condition\_variable.
    \end{miniItemize}
}


% 2018 Fall
\newcommand{\TSP}{
    \poswithprd{Course Project: Travelling salesman problem}{2018.12}
    \begin{miniItemize}
        \item
        Designed and implemented a heuristic algorithm and an exact algorithm for the travelling salesman problem using C++.
        \item
        Implemented Prim's MST algorithm.
        % Applied branch and bound for the exact algorithm:
        % Use the output of the heuristic algorithm as the initial upper bound.
        In the exact algorithm, used the algorithm to search for the MST that spans the unvisited points to estimate the cost of the current branch. Prune the current branch if the estimation exceeds the minimal cost so far.
    \end{miniItemize}
}
\newcommand{\SillyQL}{
    \poswithprd{内存数据库}{2018.11}
    \begin{miniItemize}
        \item
        用C++实现一个基于内存的关系型数据库. 支持非嵌套的INSERT, DELETE, SELECT, 与JOIN.
        \item
        支持在任意列上建立哈希索引与树索引. 通过仅在使用索引前更新索引, 避免非必要计算以提高性能.
    \end{miniItemize}
}
\newcommand{\LCTwoK}{
    \poswithprd{LC2K指令集}{2018.09-2018.12}
    \begin{miniItemize}
        \item
        用C语言实现LC2K指令集的汇编器与链接器.
        \item
        用C语言实现LC2K指令集的5级流水线模拟器. 支持简单的分支预测与自定义缓存.
    \end{miniItemize}
}

% SJTU projects
\newcommand{\HACKxSJTU}{
    \poswithprd{HACKxSJTU}{2018.6}
    \begin{miniItemize}
        \item
        Designed an AR navigation application, which places pets at crossings to indicate  directions.
        \item
        Used Unity and ARCore to implement the application on Android devices. Used C\# to control the motion of the pets.
    \end{miniItemize}
}
\newcommand{\RoboticsCompetition}{
    \poswithprd{上海交通大学新生机械创意大赛}{2017.03-2017.04}
    \begin{miniItemize}
        % TODO: organize the words better
        \item
        设计一个可以搬运并精确抓取放置各种物品的机器人. 赢得比赛冠军.
        \item
        负责机器人的框架设计. 与全队成员保持密切沟通以确保框架符合所有功能需求.
    \end{miniItemize}
}

% JI freshman courses
\newcommand{\ParkingLots}{
    \poswithprd{停车场系统}{2017.07-2017.08}
    \begin{miniItemize}
        \item
        领导4人小组. 通过OpenGL绘制动态停车场动画.
        \item
        负责所有类接口的设计与基类的实现.
    \end{miniItemize}
}
\newcommand{\TowerDefence}{
    \poswithprd{Course Project: Tower Defense Game}{Sept 2016-Oct 2016}
    \begin{miniItemize}
        \item
        Designed a robotic car, which could run along the track, start and stop under conditions specified by the rules.
        \item
        Designed a tower, which used distance sensors to locate the robotic cars and used a LASER to shoot the cars.
        \item
        Implemented the car and the tower, using Arduino as the control unit.
    \end{miniItemize}
}
